% !TeX root=main.tex
% در این فایل، عنوان پایان‌نامه، مشخصات خود، متن تقدیمی‌، ستایش، سپاس‌گزاری و چکیده پایان‌نامه را به فارسی، وارد کنید.
% توجه داشته باشید که جدول حاوی مشخصات پروژه/پایان‌نامه/رساله و همچنین، مشخصات داخل آن، به طور خودکار، درج می‌شود.
%%%%%%%%%%%%%%%%%%%%%%%%%%%%%%%%%%%%
% دانشگاه خود را وارد کنید
\university{علم و صنعت ایران}
% دانشکده، آموزشکده و یا پژوهشکده  خود را وارد کنید
\faculty{دانشکده مهندسی کامپیوتر}
% گروه آموزشی خود را وارد کنید
% \department{گروه هوش مصنوعی و رباتیک}
% گروه آموزشی خود را وارد کنید
\subject{مهندسی کامپیوتر}
% گرایش خود را وارد کنید
%\field{هوش مصنوعی و رباتیک}
% عنوان پایان‌نامه را وارد کنید
\title{افزایش تست‌پذیری برنامه با اعمال ریفکتورینگ خودکار روی کد برنامه}
% نام استاد(ان) راهنما را وارد کنید
\firstsupervisor{سعید پارسا}
% \secondsupervisor{استاد راهنمای دوم}
% نام استاد(دان) مشاور را وارد کنید. چنانچه استاد مشاور ندارید، دستور پایین را غیرفعال کنید.
% \firstadvisor{استاد مشاور اول}
%\secondadvisor{استاد مشاور دوم}
% نام دانشجو را وارد کنید
\name{صادق}
% نام خانوادگی دانشجو را وارد کنید
\surname{جعفری}
% شماره دانشجویی دانشجو را وارد کنید
\studentID{97521189}
% تاریخ پایان‌نامه را وارد کنید
\thesisdate{شهریور 1401}
% به صورت پیش‌فرض برای پایان‌نامه‌های کارشناسی تا دکترا به ترتیب از عبارات «پروژه»، «پایان‌نامه» و »رساله» استفاده می‌شود؛ اگر  نمی‌پسندید هر عنوانی را که مایلید در دستور زیر قرار داده و آنرا از حالت توضیح خارج کنید.
%\projectLabel{پایان‌نامه}

% به صورت پیش‌فرض برای عناوین مقاطع تحصیلی کارشناسی تا دکترا به ترتیب از عبارات «کارشناسی»، «کارشناسی ارشد» و »دکترا» استفاده می‌شود؛ اگر  نمی‌پسندید هر عنوانی را که مایلید در دستور زیر قرار داده و آنرا از حالت توضیح خارج کنید.
%\degree{}

\firstPage
\besmPage
\davaranPage

%\vspace{.5cm}
% در این قسمت اسامی اساتید راهنما، مشاور و داور باید به صورت دستی وارد شوند
%\renewcommand{\arraystretch}{1.2}
\begin{center}
\begin{tabular}{| p{8mm} | p{18mm} | p{.17\textwidth} |p{14mm}|p{.2\textwidth}|c|}
\hline
ردیف	& سمت & نام و نام خانوادگی & مرتبه \newline دانشگاهی &	دانشگاه یا مؤسسه &	امضـــــــــــــا\\
\hline
۱  &	استاد راهنما				 & دکتر \newline سعید پارسا & دانشیار & دانشگاه \newline علم و صنعت ایران &  \\
\hline
۲ &     استاد مشاور				 & دکتر \newline مهرداد آشتیانی & استادیار & دانشگاه \newline علم و صنعت ایران & \\
\hline
%۳ &      استاد مدعو\newline  خارجی			 & دکتر \newline محمدحسن \newline قاسمیان & استاد & دانشگاه \newline تربیت مدرس & \\
%\hline
%۴ &	استاد مدعو\newline  خارجی			 & دکتر \newline  نصرالله مقدم & استادیار & دانشگاه \newline  تربیت مدرس& \\
%\hline
%۵ &	استاد مدعو\newline  داخلی			 & دکتر\newline  رضا برنگی & استادیار & دانشگاه \newline  علم و صنعت ایران & \\
%\hline
%۶ &	استاد مدعو\newline  داخلی			 & دکتر\newline  محسن سریانی & استادیار & دانشگاه \newline  علم و صنعت ایران & \\
%\hline
%۷ &	استاد مدعو\newline  داخلی			 &دکتر \newline محمدرضا جاهدمطلق & دانشیار& دانشگاه \newline  علم و صنعت ایران & \\
\hline
\end{tabular}
\end{center}

\esalatPage
\mojavezPage


% چنانچه مایل به چاپ صفحات «تقدیم»، «نیایش» و «سپاس‌گزاری» در خروجی نیستید، خط‌های زیر را با گذاشتن ٪  در ابتدای آنها غیرفعال کنید.
 % پایان‌نامه خود را تقدیم کنید!

 \newpage
\thispagestyle{empty}
{\Large تقدیم به:}\\
\begin{flushleft}
{\huge
%همسر و فرزندانم\\
%\vspace{7mm}
%و\\
\vspace{7mm}
پدر و مادرم
}
\end{flushleft}


% سپاس‌گزاری
\begin{acknowledgementpage}
سپاس خداوندگار حکیم را که با لطف بی‌کران خود، آدمی را زیور عقل آراست.


در آغاز وظیفه‌  خود  می‌دانم از زحمات بی‌دریغ استاد  راهنمای خود،  جناب آقای دکتر پارسا و جناب آقای مرتضی ذاکری دانشجوی دوره دکترای ایشان صمیمانه تشکر و  قدردانی کنم  که قطعاً بدون راهنمایی‌های ارزنده‌  ایشان، این رساله  به انجام  نمی‌رسید.

از جناب  آقای  دکتر آشتیانی   که زحمت  مطالعه و مشاوره‌  این رساله را تقبل  فرمودند و در آماده سازی  این رساله، به نحو احسن اینجانب را مورد راهنمایی قرار دادند، کمال امتنان را دارم.

 در پایان، بوسه می‌زنم بر دستان خداوندگاران مهر و مهربانی، پدر و مادر عزیزم و بعد از خدا، ستایش می‌کنم وجود مقدس‌شان را و تشکر می‌کنم از خانواده عزیزم به پاس عاطفه سرشار و گرمای امیدبخش وجودشان، که بهترین پشتیبان من بودند.
% با استفاده از دستور زیر، امضای شما، به طور خودکار، درج می‌شود.
\signature 
\end{acknowledgementpage}
%%%%%%%%%%%%%%%%%%%%%%%%%%%%%%%%%%%%
% کلمات کلیدی پایان‌نامه را وارد کنید
\keywords{آزمون‌پذیری کد, الگوی تزریق, الگوی کارخانه, بوی بد کد}
%چکیده پایان‌نامه را وارد کنید، برای ایجاد پاراگراف جدید از \\ استفاده کنید. اگر خط خالی دشته باشید، خطا خواهید گرفت.

\fa-abstract{
هدف از این پروژه ارائه روشی برای تایین میزان آزمون پذیری برنامه‌ها می‌باشد. برای این منظور  باید بتوان با تحلیل خود برنامه میزان وابستگی بین کلاس ها را مشخص کرد. مسلما با افزایش میزان وابستگی آزمون واحد \lr{(unit test)} کلاس ها پیچیده تر می‌گردد. در واقع این وابستگی موجب می‌شود که ردیابی خطا در داخل برنامه پیچیده شود چرا که با مشاهده خطا در یک متد در داخل یک کلاس به سادگی مشخص نمی‌شود که آیا علت خطا در آن متد است یا در متدهایی که به آن ها وابستگی وجود دارد. می‌توان با استفاده از الگوهایی مانند الگوی تزریق\lr{(injection pattern)} و الگو کارخانه \lr{(factory pattern)} و به کار‌گیری راهکارهای بازسازی\lr{(refactoring code)} این گونه اشکالات یا در اصطلاح بوی بد کد \lr{(code smell)} را از میان برداشت.
}
%\begin{enumerate}
%\item ایجاد ابزاری برای استخراج مدل ارتباطی کلاس‌ها از متن پروژه های جاوا
%\item ایجاد ابزاری برای تحلیل مدل وابستگی کلاس‌ها و محاسبه میزان آزمون پذیری کد بر اساس میزان وابستگی کلاس‌ها به یک دیگر
%\item ایجاد ابزاری برای اعمال الگوی تزریق بر کلاس‌های جاوا و در واقع تولید کلاس رابط \lr{(interface)} برای برقراری ارتباط با هر کلاس
%\item پیاده سازی الگوی کارخانه با بررسی چگونگی ارتباط بین کلاس‌ها
%\end{enumerate}


\abstractPage

\newpage\clearpage